% !TEX encoding = UTF-8 Unicode

\documentclass[a4paper]{article}

\usepackage{color}
\usepackage{url}
\usepackage[T2A]{fontenc} % enable Cyrillic fonts
\usepackage[utf8]{inputenc} % make weird characters work
\usepackage{graphicx}
\graphicspath{ {img/} }

\usepackage[english,serbian]{babel}
%\usepackage[english,serbianc]{babel} %ukljuciti babel sa ovim opcijama, umesto gornjim, ukoliko se koristi cirilica

\usepackage[unicode]{hyperref}
\hypersetup{colorlinks,citecolor=green,filecolor=green,linkcolor=blue,urlcolor=blue}

%\newtheorem{primer}{Пример}[section] %ćirilični primer
\newtheorem{primer}{Primer}[section]

\begin{document}

%\renewcommand{\abstractname}{Apstrakt} %pisace Sazetak ako se ne ukljuci ova naredba

\title{Primena mašinskog učenja u verifikaciji softvera\\ \small{Seminarski rad u okviru kursa\\Metodologija stručnog i naučnog rada\\ Matematički fakultet}}

\author{Nikola Dimitrijević, Rastko Đorđević,\\
 Luka Živanović, Dimitrije Špadijer\\
 nikoladim95@gmail.com, mi14078@alas.matf.bg.ac.rs,\\
  mi14164@alas.matf.bg.ac.rs, mm11021@alas.matf.bg.ac.rs}
%\date{9.~april 2015.}
\vspace*{-3cm}
    {\let\newpage\relax\maketitle}

\abstract{
U ovom tekstu je ukratko prikazana osnovna forma seminarskog rada. Obratite pažnju da je pored ove .pdf datoteke, u prilogu i odgovarajuća .tex datoteka, kao i .bib datoteka korišćena za generisanje literature. Na prvoj strani seminarskog rada su naslov, apstrakt i sadržaj, i to sve mora da stane na prvu stranu! Kako bi Vaš seminarski zadovoljio standarde i očekivanja, koristite uputstva i materijale sa predavanja na temu pisanja seminarskih radova. Ovo je samo šablon koji se odnosi na fizički izgled seminarskog rada (šablon koji \emph{morate} da ispoštujete!) kao i par tehničkih pomoćnih uputstava. Molim Vas da kada budete predavali seminarski rad, imenujete datoteke tako da sadrže temu seminarskog rada, kao i imena i prezimena članova grupe (ili samo temu i prezimena, ukoliko je sa imenima predugačko). Predaja seminarskih radova biće isključivo preko web forme, a NE slanjem mejla.}

\tableofcontents

\newpage



% ==============================================================================
\section{Uvod}
\label{sec:uvod}
% ==============================================================================



Računarska rešenja se koriste za svemirske rakete, medicinsku opremu, samovozeće automobile i razne druge oblasti.
Sve veća zastupljenost i složenost softvera dovodi do veće verovatnoće katastrofalnog ishoda u slučaju greške.
Privremena nedostupnost vlastitog novca, pad satelita, prekomerno doziranje terapije pacijentima i puštanje nasilnih zatvorenika
na slobodu su primeri stvarnih događaja usled grešaka u programima.
Verifikacija softvera je oblast računarstva koja se bavi analizom softvera i ispitivanjem ispravnosti
programskog koda. Postoje razni alati za verifikaciju koji se uspešno koriste, ali postoje i mnoga ograničenja
u procesu analize koda koje je teško, a neke čak i nemoguće prevazići.
Analiziranje programskog koda se zasniva na formalnim matematičkim modelima koje je teško automatizovati.
Mašinsko učenje je postalo vrlo popularno i to sa pravom jer ima velike uspehe u raznim domenima,
pa je i prirodno nalaziti primene u okviru verifikacije softvera. Ovaj rad se nastavlja na \cite{micovic}
pa ovde neće biti prikazane već pomenute primene algoritama. U nastavku je dat opis verifikacije softvera,
modela mašinskog učenja koji su dalje korišćeni u razmatranim radovima, i konačno pregled konkretnih
primena mašinskog učenja kako u statičkoj tako i u dinamičkoj verifikaciji softvera.



% ==============================================================================
\section{Verifikacija softvera}
\label{sec:verifikacija}
% ==============================================================================



\par U savremenom dobu, računari i računarski sistemi predstavlju sastavni deo svakodnevice, kako u privatnom životu, tako i u poslovnom svetu, a i u državnoj administraciji. Zato je od izuzetnog značaja da softver koji se koristi bude pouzdan. Neispravan softver, u zavisnosti od toga gde se koristio i koliko je veliki bio propust, može izazvati male, ali i ogromne probleme sa teškim posledicama (čak i smrtnim). Oblast razvoja softvera koja se bavi proverom ispravnosti softvera, odnosno potvrđivanjem da softver radi u skladu sa zahtevima, naziva se \emph{verifikacija softvera}.

\par Potrebno je precizirati šta se podrazumeva pod ispravnim softverom. Razlikuju se pojmovi potpuno ispravnog i delimično ispravnog softvera. Softver se smatra potpuno ispravnim ako se zaustavlja za svaki ulaz i na izlazu daje ispravan rezultat, dok se delimično ispravnim softverom smatra onaj koji za ulaz daje ispravan rezultat ako se zaustavlja, a dozvoljeno je da se za neki ulaz program ne zaustavlja. Ne postoji algoritamski način da se proveri da li se neki program zaustavlja (tzv. Halting problem), pa se često ni ne ispituje potpuna, već samo delimična ispravnost softvera.

\par Postoje dve osnovne vrste verifikacije softvera. To su dinamička i statička verifikacija softvera.

\subsection{Dinamička verifikacija softvera}
\label{subsec:dinamicka}

\par Dinamička verifikacija softvera zasnovana je na tome da se provera ispravnosti softvera vrši tokom njegovog izvršavanja. Najčešće se dinamička verifikacija softvera vrši testiranjem i ti pojmovi se poistovećuju, što nije sasvim ispravno.

\par Testiranje je složen proces koji obuhvata pronalaženje što raznovrsnijeg skupa ulaza, definisanje očekivanih izlaza za svaki od tih ulaza, a zatim izvršavanje programa i provera da li je program za date ulaze vratio odgovarajuće izlaze. Postoje razne vrste i nivoi testova (da li imamo pristup izvornom kodu softvera ili ne, kao i da li testiramo jednu jedinicu koda ili više njih ili testiramo neku funkcionalnost softvera).

\par Testiranje služi da nam, u slučaju kada postoji greška u izvornom kodu softvera, ukaže na njeno postojanje. U zavisnosti od vrste testa, kao i nivoa, može se otkriti gde se u kodu nalazi greška, ali to često nije slučaj. Umesto toga, testovi samo otkrivaju postojanje grešaka.

\par Važno je imati na umu da testovi mogu dokazati isključivo neispravnost softvera, ali ne i njegovu (delimičnu) ispravnost. Da bi testiranje moglo dokazati delimičnu ili potpunu ispravnost softvera, neophodno bi bilo da skup ulaza bude jednak skupu svih teoretski dozvoljenih ulaza, a to, naravno, nije moguće, jer je taj skup beskonačan. Postavlja se pitanje zašto se onda uopšte testira. Odgovor je zato što se pažljivim odabirom skupa ulaza za koji će biti testirana ispravnost izlaza ne samo otklanjaju postojeće greške, nego i stiče sigurnost, odnosno poverenje u sam softver, tj. očekuje se da će broj neotkrivenih grešaka biti zanemarljiv.

\par Kao što je rečeno, često se dinamička verifikacija softvera poistovećuje sa testiranjem, iako je testiranje samo jedan od metoda. Pored testiranja postoji i debagovanje i ono očigledno spada u vid dinamičke verifikacije softvera. Naime, debagovanjem se može prekinuti rad programa u bilo kom trenutku i utvrditi trenutno stanje programa, a samim tim i potencijalno postojanje greške. Za razliku od testiranja koje se može vršiti i bez posedovanja izvornog koda softvera, debagovanje se vrši isključivo uz posedovanje koda, a često se može pronaći i tačno mesto na kome se nalazi greška.

\subsection{Statička verifikacija softvera}
\label{subsec:staticka}

\par Za razliku od dinamičke verifikacije softvera, \textit{statička verifikacija} podrazumeva analizu ispravnosti softvera bez njegovog pokretanja. Osnovna podela statičke verifikacije je na \textit{pregled koda (engl. code review)} i na \textit{automatizovanu verifikaciju}, koja će biti ključna za ostatak rada i biće podrazumevana kada se navodi pojam statička verifikacija.

\par Kao što je već pomenuto, ne postoji način da se za svaki program utvrdi da li se on zaustavlja ili ne, tako da nam i statička verifikacija ne može uvek dati željeni odgovor. Ipak, statička verifikacija može da nam da uvid u kvalitet koda i njegove propuste za mnoge netrivijalne probleme i zbog toga se razvijaju tehnike kojim bi skup takvih problema rastao, a vreme obrade se smanjivalo. 
\\\\

Neke od bitnijih tehnika statičke verifikacije su:
\begin{itemize}
\item \textbf{Analiza toka podataka} (engl. data flow analysis)
\item \textbf{Apstraktna interpretacija} (engl. abstract interpretation)
\item \textbf{Simbolička analiza} (engl. symbolic analysis)
\item \textbf{Proveravanje ograničenih modela} (engl. Bounded model checking), koja se najviše koristi za verifikaciju logičkih kola.
\end{itemize}

% TODO opisati tehnike koje smo koristili dole

% ==============================================================================
\section{Mašinsko učenje}
% ==============================================================================



\textit{Mašinsko učenje} bavi se proučavanjem indukcije, odnosno generalizacije 
čime formalizuje uopštavanje od uzorka određene veličine ka univerzalnim zaključcima. 
U srcu novog zamaha veštačke inteligencije nalazi se oblast mašinskog učenja. 
U nekim domenima, kao na primer prepoznavanje lica, u kojima se računari nisu 
mogli porediti sa ljudima po uspešnosti, ova oblast postiže rezultate superiorne 
u odnosu na rezultate ljudskih eksperata.

% TODO mozda ubaciti ovo, izbaciti nesto drugo
% Mašinsko učenje predstavlja automatsku detekciju smislenih šablona u podacima.

Mašinsko učenje je posebno pogodno za probleme koje je čoveku teško definisati a 
izuzetno lako za rešiti i u kojima je prihvatljiva poneka greška. Zbog dopuštanja 
povremenih greški u izvršavanju, ova oblast na prvi pogled izgleda kao nepogodna 
za rešavanje problema statičke verifikacije softvera. Ipak u stanju je da doprinese 
kao dodatni alat za rešavanje raznih problema koji se mogu naći u statičkoj 
verifikaciji ako njime rukuje prava osoba, čak i ako ne može da obeća kompletno 
rešenje svih problema.

Obično se identifikuju dve oblasti mašinskog učenja: \textit{nadgledano učenje} i 
\textit{nenadgledano učenje}. Nadgledano učenje odlikuje da pored ulaznih podataka 
postoji i njemu odgovarajući željeni izlaz. Model je zadužen za učenje pravila nad 
datim parovima ulaza i izlaza. Sa druge strane, kod nenadgledanog učenja dati su 
samo ulazni podaci pri čemu na algoritmu ostaje da pronađe strukturu u podacima. 
% TODO da li je naredna recenica suvisna (ne sme pasus da ima samo jednu recenicu, zato spojih sa gornjim)
Za oblast statičke verifikacije najrelevantniji su algoritmi nadgledanog učenja, 
tako da će u ostatku rada biti predstavljeni isključivo algoritmi koji pripadaju toj oblasti.

Uobičajeni tok uspešne primene tehnika mašinskog učenja je sledeći. Nakon što se 
problem analizira, kao i podaci nad kojim će tehnike biti primenjene, potrebno je 
izabrati odgovarajući model. Nakon odabira modela, podaci se pretprocesiraju i 
izabrani model se trenira nad tako obrađenim podacima. Istrenirani model je dalje 
neophodno evaluirati kako bi bilo moguće znati u kojoj meri je koristan.

\subsection{Odabrani algoritmi}



\subsection{Evaluacija modela i mere kvaliteta}

%TODO da li je naredna rečenica suvišna? (ne sme pasus samo 1 recenicu)
%Ukratko će biti predstavljeni termini vezani za \textit{meru kvaliteta} modela, 
%kako bi bilo olakšano razumevanje postignutih rezultata u ostatku rada.

\textit{Evaluacija modela} predstavlja numeričko predstavljanje sposobnosti 
predviđanja datog modela na određenoj skali i direktno se oslanja na mere 
kvaliteta. Mere koje se najčešće sreću u klasifikaciji su \textit{tačnost}, 
\textit{preciznost}, \textit{odziv} i \textit{F1 mera}. Sve navedene mere 
izvedene su iz \textit{matrice konfuzije} \ref{table:matrica_konfuzije}.

\begin{table}[h]
	\centering
	\begin{tabular}{ |c|cc| } 
		\hline
		Stvarno / Predviđeno & Pozitivno & Negativno \\ 
		\hline
		Pozitivno & Stvarno Pozitivno & Lažno Negativno \\ 
		Negativno & Lažno Pozitivno & Stvarno Negativno \\ 
		\hline
	\end{tabular}
	\caption{Matrica konfuzije}
	\label{table:matrica_konfuzije}
	%\caption{ tabelica stagod}
\end{table}

Stvarno pozitivne i stvarno negativne instance su one instance koje su ispravno 
određene od strane modela. Lažno pozitivne instance su negativne instance za koje 
je od strane modela predviđeno da su pozitivne, dok su lažno negativne instance 
zapravo pozitivne instance koje je model loše klasifikovao kao negativne. 

% TODO dodati šta koje mere zapravo rade, zašto tačnost nije savršena, napomenuti da ima još raznih
Tačnost klasifikacije predstavlja procenat uspešno klasifikovanih instanci u odnosu 
na ukupan broj instanci. Preciznost je udeo ispravno pronađenih pozitivnih instanci 
u svim instancama koje su proglašene za pozitivne. Odziv je udeo ispravno pronađenih 
pozitivnih instanci u svim zaista pozitivnim instancama. Preciznost i odziv su dve 
mere koje najviše ima smisla razmatrati zajedno i način na koji se to najčešće radi 
je F1 mera, koja predstavlja njihovu harmonijsku sredinu.



% ==============================================================================
\section{Odabrani problemi statičke verifikacije}
\label{sec:naslovN}
% ==============================================================================



% TODO srediti ovo, odgovoriti na postavljena pitanja, dopuniti nakon što odlučimo šta će biti sva 4 problema
Predlozi za ovaj deo:

Kratak opis problema sa kojima ćemo se susreti u narednom delu poglavlja. 

Zašto su odabrani baš ovi problemi?

Predstojeći problemi su uređeni po uticaju koji su imali na oblast statičke verifikacije softvera.

Predlozi za dodatne baljezgarije kojima je mesto u ovom kutku rada su dobrodošli.


\subsection{Od izvornog koda do istreniranih modela}
\label{subsec:pregled}

Rad \cite{staticFeatures} predlaže rešenje za generisanje ulaznih podataka za algoritme mašinskog učenja na osnovu
izvornog koda kako bi se napravio inteligentni metod detekcije grešaka.

Bitno je prvo definisati kakvim tipovima grešaka se posvećuje pažnja u okviru metoda.
Iako postoje drukčije greške selekcija je vršena prema važnosti, odnosno prema tome
koliku štetu mogu da uzrokuju i mogućnost da budu detektovane računarskim alatom. Izabrani tipovi grešaka su:
% može se navesti iz koje knjige su uzete !!!!

\begin{itemize}
\item Prekoračenje bafera

\item Upravljanje memorijom

\item Dereferenciranje Null pokazivača

\item Kontrola toka

\item Konverzija označene celobrojne vrednosti u neoznačenu
\end{itemize}


Korišćen je WEKA softver koji je kolekcija algoritama mašinskog učenja.
Bogat skup funkcionalnosti koje pruža WEKA uključuje:
%lista
\begin{itemize}
\item Pretprocesiranje podataka i vizuelizacija
\item Selekcija atributa
\item Algoritmi klasifikacije
\item Algoritmi predviđanja
\item Algoritmi klasterovanja
\item Pravila asocijacije
\item Tehnike evaluacije
\end{itemize}


Za dati problem su najbitniji selekcija atributa i razni algoritmi klasifikacije i predviđanja.
Prvi prikazuju koji od brojnih elemenata ulaznog vektora su zapravo uključeni u proces donošenja odluke a koji nisu, dok
drugi klasifikuju odnosno predviđaju ispravnost na osnovu ulaza.
Međutim, najbitnija prednost koju pruža WAKA u odnosu na ostale dostupne alate je mogućnost da se koristi
jedan standardizovani format ulaza za sve raspoložive algoritme učenja što omogućava da se koristi jedan ulaz kako bi se isprobale
razne mogućnosti.

Ulaz koji koristi WAKA je u formatu ARFF.
ARFF datoteke se sastoje od nabrajanja atributa odnosno njihovih imena i tipova, a potom navođenja svih instanci koje se prosleđuju algoritmima.

Glavni problem koji se rešava se može podeliti u tri potproblema koji su:
%list
\begin{enumerate}
\item Kako transofmisati izvorni kod u odgovarajući format za klasifikatore?
\item Kako trenirati te klasifikatore i koje podatke treba koristiti?
\item Koje su karakteristike potrebne algoritmu i koji je najbolji za dati problem?
\end{enumerate}



Odgovor na prvo pitanje je softver ``CMore'', program koji može da pretvori izvorni kod programskog
jezika C u ARFF datoteku. Glavna ideja je da treba da bude u mogućnosti da prati tok izvršavanja programa
i uhvati stanja svih uključenih promenljivih. Pošto su zapamćena sva stanja moguće je otkriti više vrsta
grešaka pristupa memoriji.

Prvi deo izvršavanja je učitavanje datoteke u memoriju i njena obrada kako bi se izvukli
razni elementi i ubacili u skladište nazvano ``Mozak''.
Mozak sadrži listu svih poznatih tipova, struktura, konstanti i globalnih promenljivih unutar analizirane datoteke,
ali najbitnije je što pamti sve funkcije i njihove parametre.

U drugom delu se prati tok izvršavanja i određuje
priroda svake naredbe, što bi moglo biti dodela, deklaracija, poziv funkcije ili nešto drugo.
Sve vreme mora da se motri na sve sakupljene informacije i da se po potrebi dopisuju instance na kraj ARFF fajla
(koji je ulaz za algoritme mašinskog učenja).


Za treniranje je potreban veliki broj različitih izvornih kodova, a pritom je potrebno obeležiti greške i imati verzije koda kod kojih je uklonjena greška.
Izvorni kodovi od kojih se prave ulazi za treniranje su pokupljeni sa javno dostupnih repozitorijuma, što je dvostruko zgodno:
velika količina datoteka se skupi na taj način, a pritom je moguće dobiti primere sa greškom i bez greške tako što se posmatraju različite verzije datoteka.

Na prvi pogled je dovoljno uzeti izlaz programa CMore u ARFF formatu za datoteku sa greškom i bez nje, ali on nema uvid u to šta je greška nego analizira celu datoteku i zato
generiše izlaze koji mogu imati više hiljada linija od kojih je većina nastala od ispravnih delova koda.
Potrebno je izdvojiti samo one linije koda koje su dovele do greške i njihove parove bez te greške.
Za tu svrhu se koristi program ``Holmes'' koji upoređuje dve ARFF datoteke i na izlazu se dobiju dve nove ARFF datoteke koje sadrže samo potrebne  instance.

Nakon što se svi modeli mašinskog učenja istreniraju vrši se analiza pogodnosti
koja vraća listu pogodnih modela tako što analizira njihovu preciznost, stopu lažno pozitivnih i lažno negativnih.
U nastavku su prikazani rezultati poređenja 71 klasifikatora, a na slikama su prikazani samo oni koji su se najbolje pokazali. Cela procedura je urađena dva puta, s tim što je drugi put umanjen broj atributa uključenih u proces.

Među 7 najboljih klasifikatora sa slike \ref{fig:acc} se nalaze tri zasnovana na najbližim susedima (``Ib1'', ``Ibk'', ``NNge''), tri zasovana na stablima
(``LMT'', ``RotationForest'' i ``ADABoost'' primenjen na ``BFTree'') i jedan zasnovan na neuronskim mrežama (``MultiLayerPerceptron'').
Na slici  \ref{fig:falsePos} se vidi isti trend kao i na slici  \ref{fig:acc}: smanjenjem broja atributa algoritmi zasnovani na stablima se lošije ponašaju dok algoritmi najbližih suseda i ``MultiLayerPerceptron'' daju bolje rezultate.

\begin{figure}[h!]
\centering
\includegraphics[width=\textwidth]{accuracy.png}
\caption{Preciznosti algoritama - sa svim atributima i bez}
\label{fig:acc}
\end{figure}



\begin{figure}[h!]
\centering
\includegraphics[width=\textwidth]{false_positive.png}
\caption{Lažno pozitivni - sa svim atributima i bez}
\label{fig:falsePos}
\end{figure}

Prema rezultatima dobijenim u  \cite{baca} vrlo je bitno da alat statičke analize ima malo lažno pozitivnih jer
u slučaju mnogo pogrešnih uzbuna programeri počinju da ignorišu te signale, ili još gore: kako bi izbegli upozorenja alata menjaju ispravan kod što dovodi do potencijalno
 novih greškaka. Na slici \ref{fig:falseNeg} se vidi da je ``NNge'' najbolji, što je i očekivano s obzirom na prilično
loše performanse što se tiče lažno pozitivnih.

\begin{figure}[h!]
\centering
\includegraphics[width=\textwidth]{false_negative.png}
\caption{Lažno negativni - sa svim atributima i bez}
\label{fig:falseNeg}
\end{figure}

\subsection{Formalna verifikacija softvera}

\par Verifikaciji softvera može se pristupiti na strogo formalan matematički način. U ovom radu biće izložen pristup o kome se detaljnije može pročitati u radu \cite{verify}.

\par Jedan od mogućih pristupa verifikaciji softvera, odnosno proveri ispravnosti softvera jeste sledeći. Ako je $T$ skup testova, onda je program $P$ ispravan u odnosu na datu specifikaciju $S$ ako i samo ako važi
$$(\forall t\in T)\ [prolazi(P,t)\wedge S\models t]\Rightarrow P\equiv S,$$
gde $prolazi(P,t)$ označava da program $P$ uspešno prolazi test $t$, $S\models t$ označava da je $t$ semantička posledica specifikacije $S$, odnosno da je test u skladu sa specifikacijom programa, a $P\equiv S$ označava da se program $P$ smatra ekvivalentnim specifikaciji $S$. Drugim rečima, program je ispravan u skladu sa specifikacijom $S$ ako i samo ako se iz činjenice da program zadovoljava svaki test koji je u skladu sa specifikacijom sledi da se program smatra ekvivalentnim specifikaciji.

\par Jedan od problema jeste pronaći skup testova $T$ za koji važi prethodna implikacija. Takav skup testova naziva se \emph{adekvatnim} skupom testova. Naravno, trivijalan primer je iscrpan skup testova, tj. skup testova koji obuhvataju sve moguće ulaze. No, u praksi, taj skup je obično beskonačan, te je kao takav neupotrebljiv. Potrebno je da se skup testova ograniči. Jedan od pristupa jeste da se nekom apriori analiza specifikacije izvuče odgovarajući skup testova. Međutim, postoji mogućnost da se dobiju testovi koje nije moguće direktno proveriti u datom programu, što dovodi do drugog tipičnog problema.

\par Problem pravljenja efikasne procedure da se proveri tačnost konjunkcije $prolazi(P,t)\wedge S\models t$ poznat je pod nazivom \emph{orakl problem}. Nije uvek lako proveriti tačnost od $prolazi(P,t)$, a pošto su program i njegova specifikacija zasnovani na drugačijim formalizmima, nije lako proveriti 
%dovrsicu sutra, ne mogu vise...

\subsection{Problem 4 - Rastko}

Ovde pišem tekst.
Ovde pišem tekst.
Ovde pišem tekst.
Ovde pišem tekst.
Ovde pišem tekst.

% ==============================================================================
\section{Primena mašinskog učenja u dinamičkoj verifikaciji}
\label{sec:dinamcikaPrimena}
% ==============================================================================


% ================================== LUKA ======================================
\subsection{Onlajn testiranje reinforcement learningom } % ONLINE REINFORCEMENT LEARNING
%\label{subsec:pregled}
% ==============================================================================


Programi koji su reaktivni, naročito ako su distribuirani, mogu se ponašati nedeterministički. Neki faktori, poput redosleda završavanja niti nisu pod potpunom kontrolom testera, ali se njihovo ponašanje može modelovati stablom odlučivanja. Ipak, veličina tog stabla može da postane izuzetno velika za pretragu, pa se pribegava tehnikama onlajn testiranja. %ONLINE

Onlajn testiranje može biti pogodnije od oflajn testiranja za reaktivne sisteme zato što se može dinamički prilagoditi za vreme izvršavanja tako da pokrije najrelevantnija ponašanja, umesto svih mogućih, što je obično nemoguće. Onlajn testiranje spada u skup testiranja baziranih na modelu (engl. model-based testing), gde tester koristi model da detektuje razlike između impelementacije pod testom (engl. implementation under test (IUT)) i modela. % ONLINE OFFLINE IUT

Razlikovaćemo poteze testera od IUTa. Akcije testera ćemo nazivati kontrolisane akcije, dok ćemo akcije IUTa nazivati posmatrane akcije. Greška usklađenosti nastaje kada IUT odbije kontrolisanu akciju modela ili kada model odbije posmatranu akciju IUTa. % IUTa

Stanja modela su memorijske lokacije programa koje su preslikane u konačni skup vrednosti. 
Pravila ažuriranja su konačan skup funkcija koje preslikavaju vrednosti iz domena stanja u domen stanja. Pravilo ažuriranja $p$ može biti parametrizovano parametrima $\bar{x}$.
Instanciranje p[$\bar{x}$] ulaznim vrednostima $\bar{v}$ odgovarajućeg tipa, obeležava se p[$\bar{v}$].
Generalno, pravila ažuriranja mogu biti nedeterministička, tako da za isto ulazno stanje i vrednosti, može da ima različita izlazna stanja. Da bismo to modelovali, definišemo relaciju $[[p]] \subseteq Stanja \times Vrednost^n \times Stanja$.
Kada je p determinističko, [[p]] smatramo funkcijom [[p]]: $Stanja \times Vrednost^n  \rightarrow Stanja$

Čuvar $\varphi$ je formula koja zavisi od stanja. Može sadržati slobodne logičke promenljive $\bar{x} = x_1, x_2,..., x_n$ i takvo označavamo sa $\varphi(\bar{x})$. $\varphi$ je zatvoreno ako nema slobodnih varijabli. Zatvorena formula $\varphi$ ima istinitosnu interpretaciju u stanju $s \models \varphi$. Čuvano pravilo ažuriranja je par $(\varphi, p)$ i ono limitira primenu p na stanja i argumente koji zadovoljavaju $\varphi(\bar{v})$.


Model programa P sadrži sledeće komponente
\begin{itemize}
\item Prostor stanja $Stanja$
\item Prostor vrednosti $Vrednosti$
\item Inicijalno stanje $s_0 \in Stanja$
\item Konačni skup simbola akcija $\Sigma$ podeljen u dva disjunktna podskupa
	\item $\Sigma_c$ kontrolisanih akcija i
	\item $\Sigma_o$ posmatranih akcija
\item Familija $(\varphi_f, p_f) f \in \Sigma$ čuvanih pravila ažuriranja
	\item Arnost funkcije f je broj ulaznih parametara $p_f$
	\item Arnost reset pravila je 0 i $[[p_{Reset}]](s) = s_0$ za sve $s \models \varphi_{Reset}$
\end{itemize}
P je determinističko, ako za svako $f \in \sigma, p_f$ je determinističko.


Akcija $f$ ima oblik $f(v_1,...,v_n)$, gde je $f$ $n$-arni akcijski simbol a svako $v_i$ vrednost koja odgovara ulaznom parametru $p_f$. Kažemo da je akcija $f(\bar{v})$ dozvoljena u stanju $s$ ako $s \models \varphi(\bar{v})$. Primetimo dva specijanla slučaja: kada je reset uvek zabranjen $(\varphi_{Reset} = false)$, kada je definicija $p_{Reset}$ irelevantna, i slučaj kada je reset uvek dozvoljen $(\varphi_{Reset} = false)$, i tada $p_{Reset}$ mora da bude definisana tako da iz svakog stanja može da uspostavi početno stanje.

I IUT i model posmatramo kao interfejs automate (engl. interface automata) kako bismo formalno uspostavili vezu između njih [?]. %https://users.soe.ucsc.edu/~luca/papers/01/FSE01.pdf 
Interfejs automat M ima sledeće komponente:
\begin{itemize}
\item Skup stanja $S$
\item Neprazan podskup početnih stanja $S_init$ skupa $S$
\item Disjunktni skupovi kontrolisanih akcija $A_c$ i posmatranih akcija $A_o$
\item Dozvoljavajuće funkcije $\Gamma_c$ i $\Gamma_o$ podskupova $A_c$ i $A_o$, redom.
\item Prelazna funkcija $\sigma$ koja slika ulazno stanje i dozvoljavajuću akciju u izlazno stanje
\end{itemize}

Neka je P deterministčki model $(Stanja, Vrednosti, s_0, \Sigma, \Sigma_c, \Sigma_0, Reset, (\varphi_f, p_f)f \in \Sigma)$. Prema prethodnoj definiciji, intefejs automat P izgleda:


\begin{itemize}
\item $S_{[[P]]} = States$
\item $S^{init}_{[[P]]} = \{s_0\}$
\item $A^c_{[[P]]} = \{ f(\bar{v}) | f \in \Sigma^c, \bar{v} \subseteq Values \}$ 
\item $A^o_{[[P]]} = \{ f(\bar{v}) | f \in \Sigma^o, \bar{v} \subseteq Values \}$
\item $\Gamma^c_{[[P]]} = \{ f(\bar{v}) \in A^c_{[[P]]} | s \models \varphi_f(\bar{v}) \}$ 
\item $\Gamma^o_{[[P]]} = \{ f(\bar{v}) \in A^o_{[[P]]} | s \models \varphi_f(\bar{v}) \}$
\item $\sigma_{[[P]]}(s, f(\bar{v})) = [[P_f]](s, \bar{v}) (for f \in \Sigma, s \in States, s \models \varphi_f(\bar{v}))$
\end{itemize}

Algoritam
IUT I i model M, su oba predstavljeni kao interfejs automati.
Kako smo formalno definisali strukture modela, ostaje još da opišemo način izbora naredne akcije u algoritmu.
Ukoliko nema mogućih akcija - prekini trenutni test. 
Ukoliko nema mogućeg izbora za kontrolisanu akciju - vrati posmatranu.
Ukoliko nema mogućeg izbora za posmatranu akciju - vrati kontrolisanu.
Inače, izaberi između kontrolisane i posmatrane akcije.

Algoritam prati težine transakcija u trenutnom testu. Sledeća kontrolisana akcija se bira iz nepraznog skupa dozvoljenih akcija koristeći datu strategiju:

Za ovaj problem koristićemo algoritam mašinskog učenja reinforcement learning, o kome su osnove date ranije u radu. % OSNOVE RL
Cena testa se povećava srazmerno broju koraka kako bi se pokrio zadati procenat svih ponašanja. Iscrpno pokrivanje svih ponašanja je najčešće nemoguće. Umesto da posmatramo minimizaciju cene, možemo pretpostaviti da imamo fiksne resurse i da maksimizujemo procenat pokrivenih ponašanja. Bitno je da minimizujemo broj vraćanja u početno stanje (engl. backtracking), zato što to može biti skupa operacija. %ONLINE

Odabir sledeće akcije:

Najjeftinija: 
Biramo akciju koja najmanje košta, odnosno ima najveću nagradu.

Nasumična: 
Biramo akciju na nasumičan način

Najmanje frekfentna: 
Klasičnu implementaciju familije algoritama reinforcement learning modifikujemo idejom anti-ant algoritma [?] kako bi se izbeglo generisanje generacija redudantnih testova. %RL
Akciju biramo nasumično, ali najveću verovatnoću ima stanje koje je inverzno proporcionalno njegovoj ceni. To znači da su manje frekfentne akcije favorizovane.
Drugim rečima, ako potencijalne sledeće akcije obeležimo sa $(a_i)_{i < k}$ za neko k, i cene $(c_i)_{i < k}$, onda će verovatnoća za izbor akcije $a_i$ biti $c_i^{-1} / \sum_{j!=i}^{} c_j^{-1}$.
%procitatj vise anti-ant

U radu [?] se mogu videti primeri ekspermineta sa sva tri pristupa. % https://www.microsoft.com/en-us/research/wp-content/uploads/2016/02/online.pdf


% ==============================================================================
\section{Zaključak}
\label{sec:zakljucak}
% ==============================================================================



Ovde pišem zaključak.
Ovde pišem zaključak.
Ovde pišem zaključak.
Ovde pišem zaključak.
Ovde pišem zaključak.
Ovde pišem zaključak.
Ovde pišem zaključak.
Ovde pišem zaključak.
Ovde pišem zaključak.
Ovde pišem zaključak.
Ovde pišem zaključak.
Ovde pišem zaključak.



\addcontentsline{toc}{section}{Literatura}
\appendix
\bibliography{seminarski}
\bibliographystyle{plain}



\end{document}

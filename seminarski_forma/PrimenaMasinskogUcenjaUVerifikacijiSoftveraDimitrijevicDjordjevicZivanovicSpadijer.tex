% !TEX encoding = UTF-8 Unicode

\documentclass[a4paper]{article}

\usepackage{color}
\usepackage{url}
\usepackage[T2A]{fontenc} % enable Cyrillic fonts
\usepackage[utf8]{inputenc} % make weird characters work
\usepackage{graphicx}

\usepackage[english,serbian]{babel}
%\usepackage[english,serbianc]{babel} %ukljuciti babel sa ovim opcijama, umesto gornjim, ukoliko se koristi cirilica

\usepackage[unicode]{hyperref}
\hypersetup{colorlinks,citecolor=green,filecolor=green,linkcolor=blue,urlcolor=blue}

%\newtheorem{primer}{Пример}[section] %ćirilični primer
\newtheorem{primer}{Primer}[section]

\begin{document}

%\renewcommand{\abstractname}{Apstrakt} %pisace Sazetak ako se ne ukljuci ova naredba

\title{Primena mašinskog učenja u verifikaciji softvera\\ \small{Seminarski rad u okviru kursa\\Metodologija stručnog i naučnog rada\\ Matematički fakultet}}

\author{Nikola Dimitrijević, Rastko Đorđević,\\
 Luka Živanović, Dimitrije Špadijer\\
 nikoladim95@gmail.com, mi14078@alas.matf.bg.ac.rs,\\
  mi14164@alas.matf.bg.ac.rs, mm11021@alas.matf.bg.ac.rs}
%\date{9.~april 2015.}
\vspace*{-3cm}
    {\let\newpage\relax\maketitle}

\abstract{
U ovom tekstu je ukratko prikazana osnovna forma seminarskog rada. Obratite pažnju da je pored ove .pdf datoteke, u prilogu i odgovarajuća .tex datoteka, kao i .bib datoteka korišćena za generisanje literature. Na prvoj strani seminarskog rada su naslov, apstrakt i sadržaj, i to sve mora da stane na prvu stranu! Kako bi Vaš seminarski zadovoljio standarde i očekivanja, koristite uputstva i materijale sa predavanja na temu pisanja seminarskih radova. Ovo je samo šablon koji se odnosi na fizički izgled seminarskog rada (šablon koji \emph{morate} da ispoštujete!) kao i par tehničkih pomoćnih uputstava. Molim Vas da kada budete predavali seminarski rad, imenujete datoteke tako da sadrže temu seminarskog rada, kao i imena i prezimena članova grupe (ili samo temu i prezimena, ukoliko je sa imenima predugačko). Predaja seminarskih radova biće isključivo preko web forme, a NE slanjem mejla.

\tableofcontents

\newpage

\section{Uvod}
\label{sec:uvod}

Ко жели, може да пише рад ћирилицом. У том случају, неопходно је да су инсталирани одговарајући пакети: texlive-fonts-extra, texlive-latex-extra, texlive-lang-cyrillic, texlive-lang-other. \\

Uz sve novouvedene termine u zagradi naglasiti od koje engleske reči termin potiče. Naredni primeri ilustruju način uvođenja enlegskih termina kao i citiranje.

\begin{primer}
Problem zaustavljanja (eng.~{\em halting problem}) je neodlučiv \cite{haltingproblem}.
\end{primer}

\begin{primer}
Za prevođenje programa napisanih u programskom jeziku C može se koristiti GCC kompajler \cite{gcc}.
\end{primer}

\begin{primer}
 Da bi se ispitivala ispravost softvera, najpre je potrebno precizno definisati njegovo ponašanje \cite{laski2009software}. 
\end{primer}

Reference koje se koriste u ovom tekstu zadate su u datoteci {\em seminarski.bib}. Prevođenje u pdf format u Linux okruženju može se uraditi na sledeći način:
\begin{verbatim}
pdflatex TemaImePrezime.tex 
bibtex TemaImePrezime.aux 
pdflatex TemaImePrezime.tex 
pdflatex TemaImePrezime.tex 
\end{verbatim}
Prvo latexovanje je neophodno da bi se generisao {\em .aux} fajl. {\em bibtex} proizvodi odgovarajući {\em .bbl} fajl koji se koristi za generisanje literature. 
Potrebna su dva prolaza (dva puta pdflatex) da bi se reference ubacile u tekst (tj da ne bi ostali znakovi pitanja umesto referenci). Dodavanjem novih referenci potrebno je ponoviti ceo postupak.  


Broj naslova i podnaslova je proizvoljan. Neophodni su samo Uvod i Zaključak. Na poglavlja unutar teksta referisati se po potrebi. 
\begin{primer}
U odeljku \ref{sec:naslov1} precizirani su osnovni pojmovi, dok su zaključci dati u odeljku \ref{sec:zakljucak}.
\end{primer}

Još jednom da napomenem da nema razloga da pišete:
\begin{verbatim}
\v{s} i \v{c} i \'c ...
\end{verbatim}
Možete koristiti srpska slova
\begin{verbatim}
š i č i ć ... 
\end{verbatim}


Ovde pišem uvodni tekst.
Ovde pišem uvodni tekst. 
Ovde pišem uvodni tekst. 
Ovde pišem uvodni tekst. 


\section{Verifikacija softvera}
\label{sec:verifikacija}

\par U savremenom dobu, računari i računarski sistemi predstavlju sastavni deo svakodnevice, kako u privatnom životu, tako i u poslovnom svetu, a i u državnoj administraciji. Zato je od izuzetnog značaja da softver koji se koristi bude pouzdan. Neispravan softver, u zavisnosti od toga gde se koristio i koliko je veliki bio propust, može izazvati male, ali i ogromne probleme sa teškim posledicama (čak i smrtnim). Oblast razvoja softvera koja se bavi proverom ispravnosti softvera, odnosno potvrđivanjem da softver radi u skladu sa zahtevima, naziva se \emph{verifikacija softvera}.

\par Potrebno je precizirati šta se podrazumeva pod ispravnim softverom. Razlikujemo potpuno ispravan i delimično ispravan softver. Softver se smatra potpuno ispravnim ako se zaustavlja za svaki ulaz i na izlazu daje ispravan rezultat, dok se delimično ispravnim softverom smatra onaj koji za ulaz daje ispravan rezultat ako da se zaustavlja, a dozvoljeno je da se za neki ulaz program ne zaustavlja. Ne postoji algoritamski način da se proveri da li se neki program zaustavlja (tzv. Halting problem), pa se često ni ne ispituje potpuna, već samo delimična ispravnost softvera.

\par Postoje dve osnovne vrste verifikacije softvera. To su dinamička i statička verifikacija softvera.

\subsection{Dinamička verifikacija softvera}
\label{subsec:dinamicka}

\par Dinamička verifikacija softvera zasnovana je na tome da se provera ispravnosti softvera vrši tokom njegovog izvršavanja. Najčešće se dinamička verifikacija softvera vrši testiranjem i ti pojmovi se poistovećuju, što nije sasvim ispravno.

\par Testiranje je složen proces koji obuhvata pronalaženje što raznovrsnijeg skupa ulaza, definisanje očekivanih izlaza za svaki od tih ulaza, a zatim izvršavanje programa i provera da li je program za date ulaze vratio odgovarajuće izlaze.

\subsection{Statička verifikacija softvera}
\label{subsec:staticka}

Ovde pišem tekst. 
Ovde pišem tekst. 
Ovde pišem tekst. 
Ovde pišem tekst. 
Ovde pišem tekst. 
Ovde pišem tekst. 

\section{Drugi naslov}
\label{sec:naslov2}

Ovde pišem tekst. 
Ovde pišem tekst. 
Ovde pišem tekst. 
Ovde pišem tekst. 

\subsection{... podnaslov}
\label{subsec:podnaslovN}

Ovde pišem tekst. 
Ovde pišem tekst. 
Ovde pišem tekst. 
Ovde pišem tekst. 
Ovde pišem tekst. 
Ovde pišem tekst. 

\section{$n$-ti naslov}
\label{sec:naslovN}

Ovde pišem tekst. 
Ovde pišem tekst. 
Ovde pišem tekst. 
Ovde pišem tekst. 
Ovde pišem tekst. 

\subsection{... podnaslov}
\label{subsec:podnaslovK}

Ovde pišem tekst. 
Ovde pišem tekst. 
Ovde pišem tekst. 
Ovde pišem tekst. 
Ovde pišem tekst. 

\subsection{... podnaslov}
\label{subsec:podnaslovM}

Ovde pišem tekst. 
Ovde pišem tekst. 
Ovde pišem tekst. 
Ovde pišem tekst. 
Ovde pišem tekst. 

\section{Poslednji naslov}
\label{sec:naslovM}

Ovde pišem tekst. 
Ovde pišem tekst. 
Ovde pišem tekst. 
Ovde pišem tekst. 
Ovde pišem tekst. 
Ovde pišem tekst. 
Ovde pišem tekst. 
Ovde pišem tekst. 
Ovde pišem tekst. 

\section{Zaključak}
\label{sec:zakljucak}

Ovde pišem zaključak. 
Ovde pišem zaključak. 
Ovde pišem zaključak. 
Ovde pišem zaključak. 
Ovde pišem zaključak. 
Ovde pišem zaključak. 
Ovde pišem zaključak. 
Ovde pišem zaključak. 
Ovde pišem zaključak. 
Ovde pišem zaključak. 
Ovde pišem zaključak. 
Ovde pišem zaključak. 


\addcontentsline{toc}{section}{Literatura}
\appendix
\bibliography{seminarski} 
\bibliographystyle{plain}



\end{document}
